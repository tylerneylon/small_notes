% Options for packages loaded elsewhere
\PassOptionsToPackage{unicode}{hyperref}
\PassOptionsToPackage{hyphens}{url}
\PassOptionsToPackage{dvipsnames,svgnames,x11names}{xcolor}
%
\documentclass[
]{article}
\usepackage{amsmath,amssymb}
\usepackage{fancyvrb}
\usepackage{fvextra}
%\RecustomVerbatimEnvironment{verbatim}{Verbatim}{commandchars=\\\{\}}
\usepackage{lmodern}
\usepackage{bold-extra}
\usepackage{iftex}
\ifPDFTeX
  \usepackage[T1]{fontenc}
  \usepackage[utf8]{inputenc}
  \usepackage{textcomp} % provide euro and other symbols
\else % if luatex or xetex
  \usepackage{unicode-math}
  \defaultfontfeatures{Scale=MatchLowercase}
  \defaultfontfeatures[\rmfamily]{Ligatures=TeX,Scale=1}
\fi
% Use upquote if available, for straight quotes in verbatim environments
\IfFileExists{upquote.sty}{\usepackage{upquote}}{}
\IfFileExists{microtype.sty}{% use microtype if available
  \usepackage[]{microtype}
  \UseMicrotypeSet[protrusion]{basicmath} % disable protrusion for tt fonts
}{}
\makeatletter
\@ifundefined{KOMAClassName}{% if non-KOMA class
  \IfFileExists{parskip.sty}{%
    \usepackage{parskip}
  }{% else
    \setlength{\parindent}{0pt}
    \setlength{\parskip}{6pt plus 2pt minus 1pt}}
}{% if KOMA class
  \KOMAoptions{parskip=half}}
\makeatother
\usepackage{xcolor}
\setlength{\emergencystretch}{3em} % prevent overfull lines
\providecommand{\tightlist}{%
  \setlength{\itemsep}{0pt}\setlength{\parskip}{0pt}}
\setcounter{secnumdepth}{5}
\makeatletter
\@ifpackageloaded{subfig}{}{\usepackage{subfig}}
\@ifpackageloaded{caption}{}{\usepackage{caption}}
\captionsetup[subfloat]{margin=0.5em}
\AtBeginDocument{%
\renewcommand*\figurename{Figure}
\renewcommand*\tablename{Table}
}
\AtBeginDocument{%
\renewcommand*\listfigurename{List of Figures}
\renewcommand*\listtablename{List of Tables}
}
\newcounter{pandoccrossref@subfigures@footnote@counter}
\newenvironment{pandoccrossrefsubfigures}{%
\setcounter{pandoccrossref@subfigures@footnote@counter}{0}
\begin{figure}\centering%
\gdef\global@pandoccrossref@subfigures@footnotes{}%
\DeclareRobustCommand{\footnote}[1]{\footnotemark%
\stepcounter{pandoccrossref@subfigures@footnote@counter}%
\ifx\global@pandoccrossref@subfigures@footnotes\empty%
\gdef\global@pandoccrossref@subfigures@footnotes{{##1}}%
\else%
\g@addto@macro\global@pandoccrossref@subfigures@footnotes{, {##1}}%
\fi}}%
{\end{figure}%
\addtocounter{footnote}{-\value{pandoccrossref@subfigures@footnote@counter}}
\@for\f:=\global@pandoccrossref@subfigures@footnotes\do{\stepcounter{footnote}\footnotetext{\f}}%
\gdef\global@pandoccrossref@subfigures@footnotes{}}
\@ifpackageloaded{float}{}{\usepackage{float}}
\floatstyle{ruled}
\@ifundefined{c@chapter}{\newfloat{codelisting}{h}{lop}}{\newfloat{codelisting}{h}{lop}[chapter]}
\floatname{codelisting}{Listing}
\newcommand*\listoflistings{\listof{codelisting}{List of Listings}}
\makeatother
\ifLuaTeX
  \usepackage{selnolig}  % disable illegal ligatures
\fi
\IfFileExists{bookmark.sty}{\usepackage{bookmark}}{\usepackage{hyperref}}
\IfFileExists{xurl.sty}{\usepackage{xurl}}{} % add URL line breaks if available
\urlstyle{same} % disable monospaced font for URLs
\hypersetup{
  pdftitle={Summary and Analysis: So Good They Can't Ignore You},
  pdfauthor={Tyler Neylon},
  colorlinks=true,
  linkcolor={black},
  filecolor={Maroon},
  citecolor={Blue},
  urlcolor={Blue},
  pdfcreator={LaTeX via pandoc}}

\title{Summary and Analysis: So Good They Can't Ignore You}
\author{Tyler Neylon}
\date{\href{https://tylerneylon.com/a/7date/}{526.2023}}

%%%%%%%%%%%%%%%%%%%%%%%%%%%%%%%%%%%%%%%%%%%%%%%%%%%%%%%%%%%%%%%%%%%%%%%%%%%
% Begin custom, non-pandoc commands.

\newcommand{\customstrut}{\rule[-3mm]{0mm}{7.5mm}}
\newenvironment{densearray}{\begin{array}{rcl}}{\end{array}}
\newcommand{\class}[1]{}
\newcommand{\Rule}[3]{}
\newcommand{\optquad}{\quad}
\newcommand{\smallscrneg}{}
\newcommand{\smallscr}[1]{}
\newcommand{\bigscr}[1]{#1}
\newcommand{\smallscrskip}[1]{}

% I learned some things from these two links:
% https://tex.stackexchange.com/questions/145812/using-fbox-in-a-newenvironment
% https://tex.stackexchange.com/questions/120042/splitting-a-command-syntax-across-a-newenvironment-definition

\newsavebox{\mybox}
\newenvironment{myboxed}{\begin{lrbox}{\mybox}\begin{minipage}{0.98\textwidth}}{\end{minipage}\end{lrbox}\fbox{\usebox{\mybox}}}

\newcommand{\boxedstart}{\begin{myboxed}}
\newcommand{\boxedend}{\end{myboxed}}

\newcommand{\crossedouty}{\dot y \kern -4.5pt \raise 4.9pt \hbox{\(\scriptscriptstyle\diagup\)}}
\newcommand{\crossedoutone}{\dot 1 \kern -5.1pt \raise 6.6pt \hbox{\(\scriptscriptstyle\diagup\)}}
\newcommand{\crossedouttwo}{\dot 2 \kern -5.1pt \raise 6.6pt \hbox{\(\scriptscriptstyle\diagup\)}}
\newcommand{\crossedoutthree}{\dot 3 \kern -5.1pt \raise 6.6pt \hbox{\(\scriptscriptstyle\diagup\)}}
\newcommand{\crossedoutfive}{\dot 5 \kern -5.1pt \raise 6.6pt \hbox{\(\scriptscriptstyle\diagup\)}}
\newcommand{\crossedoutsix}{\dot 6 \kern -5.1pt \raise 6.6pt \hbox{\(\scriptscriptstyle\diagup\)}}
\newcommand{\crossedoutseven}{\dot 7 \kern -5.1pt \raise 6.6pt \hbox{\(\scriptscriptstyle\diagup\)}}
\newcommand{\lowerhaty}{\lower 1ex\hbox{\(\hat y\)}}
\newcommand{\lhy}{\lower 1ex\hbox{\(\hat y\)}}

\let\smallstart\iffalse
\let\smallend\fi

% End custom, non-pandoc commands.
%%%%%%%%%%%%%%%%%%%%%%%%%%%%%%%%%%%%%%%%%%%%%%%%%%%%%%%%%%%%%%%%%%%%%%%%%%%

\begin{document}
\maketitle

\newcommand{\R}{\mathbb{R}}
\newcommand{\N}{\mathbb{N}}
\newcommand{\eqnset}[1]{\left.\mbox{$#1$}\;\;\right\rbrace\class{postbrace}{ }}
\providecommand{\latexonlyrule}[3][]{}
\providecommand{\optquad}{\class{optquad}{}}
\providecommand{\smallscrneg}{\class{smallscrneg}{ }}
\providecommand{\bigscr}[1]{\class{bigscr}{#1}}
\providecommand{\smallscr}[1]{\class{smallscr}{#1}}
\providecommand{\smallscrskip}[1]{\class{smallscrskip}{\hskip #1}}

\newcommand{\mydots}{{\cdot}\kern -0.1pt{\cdot}\kern -0.1pt{\cdot}}

\newcommand{\?}{\stackrel{?}{=}}
\newcommand{\sign}{\textsf{sign}}
\newcommand{\order}{\textsf{order}}
\newcommand{\flips}{\textsf{flips}}
\newcommand{\samecycles}{\textsf{same$\\\_$cycles}}
\newcommand{\canon}{\textsf{canon}}
\newcommand{\cs}{\mathsf{cs}}
\newcommand{\dist}{\mathsf{dist}}
\renewcommand{\theenumi}{(\roman{enumi})}

{[} Formats:
\href{http://tylerneylon.com/b/newport_book/newport_book.html}{html}
\textbar{}
\href{http://tylerneylon.com/b/newport_book/newport_book.pdf}{pdf}
\(\,\){]}

I recently enjoyed reading the book \emph{So Good They Can't Ignore You}
by Cal Newport. This post summarizes the book and provides some analysis
of it. Since I like to be kind to authors, I also encourage you to
\href{https://www.amazon.com/So-Good-They-Cant-Ignore-You-audiobook/dp/B009CMO8JQ/}{buy
the book} if it appeals to you.

\hypertarget{summary}{%
\section{Summary}\label{summary}}

Cal's goal is essentially to answer the question: How should I plan my
career?

\emph{(I think it's strange that we call authors by their last name in
writing when we address people by first name in person. So I'm trying
out referring to Cal as Cal instead of as ``Newport,'' which would sound
stiff and overly-formal if I were to speak that way out loud.)}

He suggests these ideas as a framework for thinking about your career:

\begin{itemize}
\tightlist
\item
  Don't fall into the trap of following your passion as a first
  priority. \emph{(Chapters 1--2)}
\item
  Build up career capital in the form of rare and valuable skills.
  \emph{(Chapters 4--7)}
\item
  It's nice to have control over what you do in your work.

  \begin{itemize}
  \tightlist
  \item
    Don't opt for control until you have rare and valuable skills.
    \emph{(Chapters 9--10)}
  \item
    Do opt for control despite others wanting to be in charge of you.
    \emph{(Chapter 11)}
  \end{itemize}
\item
  It's nice to have a career mission.

  \begin{itemize}
  \tightlist
  \item
    Don't look for a mission before you have rare and valuable skills.
    \emph{(Chapter 13)}
  \item
    Once you have a mission, move toward it with bite-sized projects;
    ``little bets.'' \emph{(Chapter 14)}
  \item
    Once you have a mission, share your work in a way that makes people
    want to talk about it, and in a place that's conducive to sharing
    and discussion. \emph{(Chapter 15)}
  \end{itemize}
\end{itemize}

That's the book in a nutshell.

\hypertarget{book-origins}{%
\subsection{Book Origins}\label{book-origins}}

The book was published in 2012, just as Cal was about to begin his
tenure-track position as a computer science professor at Georgetown.
(Below I'll go into more thoughts along the line of: He seems a bit
early in his career to be giving career-length advice; for now I'll
focus on a more objective summary.) He had written a series of blog
posts refuting the passion hypothesis --- the mandate to follow your
dreams --- and they received a strong reception. He expanded on those
ideas into this book. Along the way, he interviewed a number of
interesting personalities, building his hypotheses on what worked or
didn't work for them.

\hypertarget{cals-writings-since-2012}{%
\subsection{Cal's Writings Since 2012}\label{cals-writings-since-2012}}

I was curious if Cal had evolved his thinking since publishing this
book. He was early in his career, and the book lands to me a little as a
hypothesis, rather than as a tested theory, especially for Cal himself.
I would have found value in Cal saying either ``yep, this has worked out
great for me,'' or perhaps ``golly, I think I should revise some of
those hypotheses.'' However, I didn't find evidence either way (which,
if you're a skeptical person, you might read as subtle evidence the
hypotheses could use some revision).

After 2012, Cal has written several books focusing on productivity and
reducing distractions. For example, he doesn't like the way email is
often used as a default means of communication, and one for which we
sometimes expect quick replies. Instead, he advocates for considering
email to by asynchronous --- no quick replies are expected --- and
suggests ideas like office hours, as designated times when someone is
expected to respond right away. I interpret this follow-up as Cal seeing
the need for hour-to-hour systems in improving his career experience.

\hypertarget{analysis}{%
\section{Analysis}\label{analysis}}

I take basically everything with a grain of salt, and this book is no
exception. I'm going to agree with ideas when they make sense to me and
match my experience.

To some degree, the ideas presented here make intuitive sense. Nothing
Cal says seems to be overtly mistaken. For example, I have often noticed
a feeling of boredom or even unhappiness when I'm about to begin certain
work projects, only to be surprised at how many interesting ideas could
arise in the course of working on the project. I've often seen
excitement show up as the \emph{result} of hard work, not as the
precursor.

At the same time, this book follows a common formula. Cal presents an
abstract idea up front, tells one or two stories that support this idea,
and then summarizes the idea again. Each chapter feels like this, and
the book as a whole follows a similar arc. I have to admit that I was
satisfied when Cal returned to the same story he opened with, of a man
who followed his passion to become a monk, but was miserable as a
result. That man left his life of disconnection and returned to the rat
race, and was happier.

I have a few problems with this formula:

\begin{itemize}
\tightlist
\item
  It's one-sided. Cal presents his ideas as in an opinion piece, trying
  to convince us the ideas are rock-solid, as opposed to the writing of
  an impartial expository writer showing us multiple perspectives.
\item
  It presents anecdotes as if they were strong scientific evidence. I'm
  skeptical about the reliability of anecdotes as arguments. While Cal
  lands as credible, this genre is rife with writing that I suspect
  embelishes or creates supporting stories.
\item
  It's redundant. I was sometimes bored re-reading the same ideas
  repeatedly.
\end{itemize}

Ok, that's somewhat harsh --- apologies to Cal if you read this. I aim
to be more critical of the genre than of this particular author.

Having complained about the genre, let's get into the content: Are the
ideas themselves useful?

\hypertarget{idea-passion-is-not-a-first-priority}{%
\subsection{Idea: Passion is not a first
priority}\label{idea-passion-is-not-a-first-priority}}

This one is easy for me to agree with. I appreciate that Cal later
clarifies that passion is not \emph{bad} so much as it should not be
your \emph{primary} motivation. I agree with this because I have many
interests, and I could just as easily enjoy solving KenKen puzzles as I
do researching consciousness-achieving algorithms, and one of those has
more career value than the other. This idea also resonates with some
well-respected career advice from Richard Hamming, who always wants
people to work on the most important problem they can make progress on;
this is another example of passion being subserviant to the value of
your work.

\hypertarget{idea-first-create-rare-and-valuable-skills}{%
\subsection{Idea: First create rare and valuable
skills}\label{idea-first-create-rare-and-valuable-skills}}

There's not much to disagree with here. I tried to think of careers in
which the primary measure of job skill was simply how long you've been
in the job. Any job like that would not be a good match for this advice.
All I can think of, however, are unexciting government or unionized
positions. And in both cases, I'm not sure how externally impressive
such a career would be. In other words, I have trouble thinking of any
counterexamples at all to the premise that a career is built on rare and
valuable skills.

\hypertarget{idea-carefully-bid-for-control-over-your-work}{%
\subsection{Idea: Carefully bid for control over your
work}\label{idea-carefully-bid-for-control-over-your-work}}

I mostly agree with this premise, although Cal positioned his stories to
suggest that you'll always have to fight for control in your job. I
don't think that's always true. I've worked for people who have trusted
me not because I insisted on autonomy but because I deliver good work,
and giving me autonomy was easier for them than micromanaging me. I've
also been a manager of people where I felt the same way, and voluntarily
gave them more autonomy, rather than waiting for them to request it. If
anything, I'd say that how much friction there is tells you more about
your employer than about yourself.

\hypertarget{idea-carefully-choose-a-mission-for-your-work}{%
\subsection{Idea: Carefully choose a mission for your
work}\label{idea-carefully-choose-a-mission-for-your-work}}

This is the most nebulous of Cal's ideas in the books, and was the least
satisfying for me to think about. Should I feel bad if I don't have a
mission and I'm far into my career? Or what if I chose a mission but
later changed direction? What if I'm already happy without a mission?

I suspect this part of Cal's thinking is likely influenced by his
pursuit of an academic career. If you're a musician, for example, you
may simply want to make good music. Perhaps there is an analogue of a
mission, such as moving a trend in a direction you feel is better, or
perhaps raising awareness of a cause you care about. But it would be a
lie to pretend that you write music \emph{because} of an external cause,
such as world hunger.

What I'm getting at is that I suspect the idea of a mission is something
that can clarify or augment a sense of meaning in your career, but is
not critically necessary. If you care about something, and want to make
a difference in the world, go for it. But that simple
desire-and-response need not be presented in the light of a career
necessity.

\hypertarget{whats-left-out}{%
\subsection{What's left out?}\label{whats-left-out}}

Although Cal does not explicitly say this, the tone of the book presents
it as a guide to happiness in terms of making large-scale career
decisions. I completely agree that building rare and valuable skills is
a critical low-level goal.

At the same time, the book lands to me as more a response to some career
mistakes than as a holistic overview of career planning. Or, if I'm
being even more honest, the book lands to me as someone who's good at
publicity taking full advantage of a few small ideas to sell a book
(sorry again, Cal!)

There are many factors that contribute to career happiness. Here are few
important ones that Cal doesn't talk much about:

\begin{itemize}
\tightlist
\item
  The positive impact on the world of your work.
\item
  Work-life balance.
\item
  Compensation.
\item
  Good relationships with your colleagues.
\item
  Learning and feeling challenged.
\item
  Safety.
\end{itemize}

(Cal does briefly mention the value of career \emph{impact} and
\emph{relatedness} --- work relationships --- but doesn't say much about
them.)

To be fair, Cal doesn't explicitly say he'll address all aspects of
finding happiness in your job. At the same time, the book's tone lands
as if it has everything figured out. I'd love to read more books that
present an interesting set of ideas in a style that's more succinct,
introspective, and with an honest assessment of those ideas existing
within a larger ecosystem of questions and complexities.

\end{document}

